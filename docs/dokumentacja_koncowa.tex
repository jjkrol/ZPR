\documentclass[11pt,a4paper]{article}

\usepackage[OT4]{polski}
\usepackage[utf8x]{inputenc}
\usepackage{ifthen}
\usepackage{geometry}
\usepackage{anysize}
\usepackage{fancyhdr}
\usepackage{sectsty}

\pagestyle{fancy}
\headheight 1cm
\marginsize{2cm}{2cm}{0.5cm}{1cm}
\lhead{ZPR - semestr 12L}
\chead{\Large{Dokumentacja wstępna}}
\rhead{Przeglądarka zdjęć}
\lfoot{\emph{Jakub Król, Maciej Suchecki, Jacek Witkowski}}

\fancyheadoffset{0cm}

\sectionfont{\large}

% komendy
\newcounter{ind}
\newcommand*{\wylicz}[2]{#1_1,#1_2,\dots,#1_{#2}} 
\newcommand*{\ndots}[1]{
\setcounter{ind}{1}
\whiledo{\value{ind}<#1}{. \stepcounter{ind}}} 

\newcommand*{\wyliczn}[2]{\{#1_1,#1_2,\dots,#1_{#2}\}} 

\begin{document}

\begin{titlepage}
\begin{center}
\vspace*{8cm}
\textsc{\Huge Przeglądarka zdjęć} \\[0.5cm]
{\small projekt realizowany w ramach zajęć z przedmiotu}\
{\small Zaawansowane Programowanie w C++}\\\
{\small na wydziale WEiTI PW}\\[1.5cm]
\textsc{\Large dokumentacja wstępna}\\[2cm]
\emph{Jakub Król, Maciej Suchecki, Jacek Witkowski}

\end{center}
\end{titlepage}

\section{Opis projektu} 
Przeglądarka zdjęć o prostym i intuicyjnym interfejsie, pozwalająca na efektywne zarządzanie zdjęciami i ich edycję. Program ma oferować przede wszystkim najpotrzebniejsze funkcje niezbędne do obróbki fotografii, rezygnując z mniej popularnych opcji na rzecz zwiększenia wydajności funkcji podstawowych. Aplikacja będzie miała dostępny interfejs umożliwiający proste dodawanie wtyczek, które pozwolą rozszerzyć oferowane funkcje, przede wszystkim pod względem edycji zdjęć.

\section{Rozwinięcie szczegółowe}
Program będzie realizował funkcje określone w następujących kategoriach:

\subsection{Przeglądanie}
\begin{itemize}
  \item widok jednego zdjęcia
  \begin{itemize}
    \item wyświetlanie zdjęć 
    \item powiększanie i zmniejszanie podglądu
    \item dopasowanie zdjęcia do wielkości okna
    \item wyświetlanie podstawowych informacji o aktualnie oglądanym pliku
  \end{itemize}
  \item widok wielu zdjęć
  \begin{itemize}
    \item możliwość wyboru przeglądanej struktury: dyskowa lub tagi
    \item wyświetlanie miniatur wielu zdjęć z zadanego obszaru w formie siatki
  \end{itemize}
\end{itemize}

\subsection{Organizacja}
\begin{itemize}
  \item wybór folderów tworzących bibliotekę
  \item organizacja wg systemu plików
  \item organizacja własna (tagi)
  \begin{itemize}
    \item dodawanie tagów 
    \item usuwanie tagów
    \item selekcja według tagów
  \end{itemize}
\end{itemize}

\subsection{Edycja}
\begin{itemize}
  \item modularna reprezentacja systemu efektów np.:
      \begin{itemize}
        \item desaturacja
        \item rotacja
        \item zmiana rozmiaru
      \end{itemize}
  \item edycja w zewnętrznym edytorze
  \item kontrola niezpisanych zdjęć i potwierdzenie zapisu przy zakończeniu pracy
\end{itemize}

\section{Założenia projektowe}
Program będzie:
\begin{itemize}
  \item zrealizowany w architekturze wielowątkowej.
  \item przenośny pomiędzy różnymi platformami 
  \item obsługiwał zdjęcia zapisane w formacie graficznym JPEG
\end{itemize}

\section{Podsumowanie}
Program pozwoli na zastosowanie i poszerzenie zdobytej na wykładzie wiedzy z zakresu programowania, jednocześnie będąc w pełni funkcjonalną aplikacją użytkową, którą będą mogli posługiwać się w przyszłości inni użytkownicy.


\end{document}
